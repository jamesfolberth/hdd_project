\documentclass[12pt]{article}
\usepackage{graphicx}
\usepackage{amsmath,amssymb}
\textheight 240mm
\textwidth  170mm
\oddsidemargin  0mm
\evensidemargin 0mm
\topmargin -20mm
\begin{document}                                       
%_________________________________________________________________
\title{Project Proposal\\
James Folberth, Dale Jennings, Alyson F0x}
\maketitle
%_________________________________________________________________
\section{Introduction}
%_________________________________________________________________
Digital music has changed the face of music collection. Users can have megabyte or a gigabyte worth of music on their personal labtops and can easily download music from the internet. This has lead to the need to invent and test new tools in musical information retrieval. There are many websites and apps, Pandora, iTunes are a few examples, that can create playlist based on similarity to a certain song or genre of music. These sites need to be able to classify certain features of a song and build the playlist from there. This project is based on being able to classify the genre of a song. This may seem like a simple problem, since we can usually classify a song by ear. However, automating the process to computers adds complexity that must be dealt with. 
%_________________________________________________________________
\section{Distance in song-space}
Explain what are the available distance in the space of songs.
Describe your distance and any pre-processing performed before
computing the distance.
%_________________________________________________________________
\section{Dimension reduction}
%_________________________________________________________________
Describe your dimension reduction technique, and justify why it 
is appropriate to use it in this context. You should explain what 
performance is expected.
%_________________________________________________________________
\section{Statiscal learning}
%_________________________________________________________________
Explain how the training data help find the genre of an unknown
song. This could be as simple as finding the closest song among all
the songs for which you know the genre. Or it could involve more
sophisticated methods.
%_________________________________________________________________
\section{Experiments}
%_________________________________________________________________
Describe the experiments, and include the confusion matrix. Discuss
the influence of the various parameters, and describe how the optimal
parameters were chosen. Include the computation time for your method.
%_________________________________________________________________
\section{Discussion}
%_________________________________________________________________
Provide a critique of the approach and discuss any potential
improvement. Discuss the ability of your approach to classify
non-classical into the five remaining genres.
\end{document}
