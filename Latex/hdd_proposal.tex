\documentclass[12pt]{article}
\usepackage{graphicx}
\usepackage{amsmath,amssymb}
\textheight 240mm
\textwidth  170mm
\oddsidemargin  0mm
\evensidemargin 0mm
\topmargin -20mm
\begin{document}                                       
%_________________________________________________________________
\title{Project Proposal\\
James Folberth, Dale Jennings, Alyson F0x}
\maketitle
%_________________________________________________________________
\section{Introduction}
%_________________________________________________________________
\indent Digital music has changed the face of music collection. Users can have megabyte or a gigabyte worth of music on their personal labtops and can easily download music from the internet. This has lead to the need to invent and test new tools in musical information retrieval. There are many websites and apps, Pandora, iTunes are a few examples, that can create playlist based on similarity to a certain song or genre of music. These sites need to be able to classify certain features of a song and build the playlist from there. This project is based on being able to classify the genre of a song. This may seem like a simple problem, since we can usually classify a song by ear, but that relies on the user having a vast knowledge of music. If we can automate the process using computers, we could find new and interesting insights that may not be obvious. However, this adds complexity that must be dealt with since we need techniques such that  the computer can ``listen'' to the song and extract features so that the genre can be identified. To do this we take  a song, which is a continuous signal and we sample at certain frequency to transform a signal into a discrete signal that is based on pitch which created by the sound pressures changes  and define it as a function of time. From the discrete signal we can define  ``features" for each song that many be able to used to classify the genre. 

\indent For our project we are only classifying within six different genres, classical, electronically, jazz/blues, metal/punk, rock/pop and world. To compute similarities between each song it is imperative that we generate features. Given 729 training tracks, we will construct features for each song and the song that we would like to classify  to create a different to be able to use $k$-means nearest neighbor method. Most features that we used were developed or used in Elias Pampalk dissertation \footnote{Computational Models of Music Similarity and their Application in Music Information Retrieval by Elias Pampal } %fix reference

\indent We will now describe the features used to be able determine similarity. 
%_________________________________________________________________
\section{Distance in song-space}
Explain what are the available distance in the space of songs.
Describe your distance and any pre-processing performed before
computing the distance.
%_________________________________________________________________
\section{Dimension reduction}
%_________________________________________________________________
Describe your dimension reduction technique, and justify why it 
is appropriate to use it in this context. You should explain what 
performance is expected.
%_________________________________________________________________
\section{Statiscal learning}
%_________________________________________________________________
Explain how the training data help find the genre of an unknown
song. This could be as simple as finding the closest song among all
the songs for which you know the genre. Or it could involve more
sophisticated methods.
%_________________________________________________________________
\section{Experiments}
%_________________________________________________________________

We have the following average cross validation matrix. 
\begin{center}
 \begin{tabular}{l| |l | l | l | l | l | l | }
&classical &electronic& jazz &punk& rock &world \\ \hline \hline
classical &63 & 1 & 2 & 0 & 1 & 10 \\ \hline 
electronics& 0 & 16 & 0 & 0 & 1 & 3 \\ \hline 
jazz& 0 & 0 & 1 & 0 & 0 & 0 \\ \hline 
punk &0 & 0 & 0 & 4 & 3 & 0 \\ \hline 
rock& 0 & 5 & 2 & 5 & 15 & 5 \\ \hline 
world&1 & 1 & 0 & 0 & 1 & 6 \\ \hline 
\end{tabular}
\end{center}
%_________________________________________________________________
\section{Discussion}
%_________________________________________________________________
Provide a critique of the approach and discuss any potential
improvement. Discuss the ability of your approach to classify
non-classical into the five remaining genres.
\end{document}
